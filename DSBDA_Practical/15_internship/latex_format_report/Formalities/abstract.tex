\thispagestyle{plain}
\begin{center}
    \Large \textbf{\uppercase{Abstract}}
\end{center}

\vspace{3\baselineskip}

\justify
\noindent
VoIP (voice over IP - that is, voice delivered using the Internet Protocol) is a
term used in IP telephony for a set of facilities for managing the delivery of voice
information using the Internet Protocol (IP). In general , this means sending voice
information in digital form in discrete packets rather than in the traditional circuit -
committed protocols of the public switched telephone network (PSTN ). A major
advantage of VoIP and Internet telephony is that it avoids the tolls charged by ordinary
telephone service
VoIP is therefore telephony using a packet-based network instead of the PSTN (
circuit switched).
During the early 90's the Internet was beginning its commercial spread . The
Internet Protocol (IP), part of the TCP/IP suite (developed by the U.S. Department of
Defense to link dissimilar computers across
many kinds of data networks) seemed to have the necessary qualities to become
the successor of the PSTN.
 The first VoIP application was introduced in 1995 - an "Internet Phone". An
Israeli company by the name of "VocalTec" was the one developing this application. The
application was designed to run on a basic PC. The idea was to compress the voice signal
and translate it into IP packets for transmission over the Internet. This "first generation"
VoIP application suffered from delays (due to congestion), disconnection , low quality (
both due to lost and out of order packets ) and incompatibility . Vocal Tec ’s Internet
phone was a significant breakthrough , although the application 's many problems
prevented it from becoming a popular product . Since this step IP telephony has
developed rapidly. The most significant development is gateway that act as an interface
between IP and PSTN network
\\

\textbf{\emph{Keywords:}} Networking, TCP Connection,
Communication through network.\\